\documentclass[a4paper,10pt]{article}
\usepackage[utf8]{inputenc}
\usepackage{amsmath,amsfonts,amssymb,graphicx,subfigure}
\usepackage{amsmath}
\usepackage{rotating}
\usepackage{lscape}
%opening
\title{SPARC: SpArse Redundant Calibration}
\author{T.L. Grobler}

\begin{document}

\maketitle

%\begin{abstract}

%\end{abstract}

\begin{section}{No Name}

\begin{equation}
\boldsymbol{H} = 
\begin{bmatrix}
\boldsymbol{A} & \boldsymbol{B}\\
\boldsymbol{B}^* & \boldsymbol{A}^*
\end{bmatrix}
\end{equation}

With 

\begin{equation}
\boldsymbol{A} = 
\begin{bmatrix}
\boldsymbol{C} & \boldsymbol{D}\\
\boldsymbol{D^H} & \boldsymbol{E}
\end{bmatrix}
\end{equation}

\begin{equation}
\boldsymbol{B} = 
\begin{bmatrix}
\boldsymbol{F} & \boldsymbol{G}\\
\boldsymbol{G}^T & \boldsymbol{0}
\end{bmatrix}
\end{equation}

\begin{equation}
[\boldsymbol{C}]_{ij} = 
\begin{cases}
 \sum_{k \neq i} \left | g_k \right |^2 \left | y_{|i-k|} \right |^2 & \textrm{if} ~ i=j\\
 0 & \textrm{otherwise}
\end{cases}
\end{equation}

\begin{equation}
[\boldsymbol{D}]_{ij} = 
\begin{cases}
 g_i y_j^*  \left | g_{i+j} \right |^2  & \textrm{if} ~ i+j \leq N\\
 0 & \textrm{otherwise}
\end{cases}
\end{equation}

\begin{equation}
[\boldsymbol{E}]_{ij} = 
\begin{cases}
 \sum_{pq \in \mathcal{I}_i} \left | g_p \right |^2 \left | g_q \right |^2  & \textrm{if} ~ i=j\\
 0 & \textrm{otherwise}
\end{cases}
\end{equation}

\begin{equation}
[\boldsymbol{F}]_{ij} = 
\begin{cases}
 g_i g_j  \left | y_{|i-j|} \right |^2  & \textrm{if} ~ i \neq j\\
 0 & \textrm{otherwise}
\end{cases}
\end{equation}

\begin{equation}
[\boldsymbol{G}]_{ij} = 
\begin{cases}
 g_i y_j  \left | g_{i-j} \right |^2  & \textrm{if} ~ i > j\\
 0 & \textrm{otherwise}
\end{cases}
\end{equation}

Moreover, 

\begin{equation}
\mathcal{I}_i = \left\{pq|(q - p) = i \right\}
\end{equation}

The dimensions of the matrices:

\begin{enumerate}
\item $\boldsymbol{H}$: is a $(4N-2)\times(4N-2)$ or a $P \times P$ matrix. $N$ denotes the number of antennas and $P$ denotes the number of parameters. This matrix
contains $6N^2 - 2N - 2$ non-zero entries. It contains $16N^2 - 16N + 4$ entries. We can construct this matrix with $4N^2 -  4N$ or $\frac{1}{4} P^2 -1$ elementary operations.
\item $\boldsymbol{A}$: is a $(2N-1)\times(2N-1)$ matrix. This matrix has $N^2 + N - 1$ non-zero entries. We can construct this matrix with $2\frac{1}{2} (N^2 -  N)$ elementary operations.  
\item $\boldsymbol{B}$: is a $(2N-1)\times(2N-1)$ matrix. This matrix has $2N^2 - 2N$ non-zero entries. We can construct this matrix with $1\frac{1}{2} (N^2 -  N)$ elementary operations.
\item $\boldsymbol{C}$: is a $N\times N$ matrix. This matrix has $N$ non-zero entries. We can construct this matrix with $N^2-N$ elementary operations.
\item $\boldsymbol{D}$: is a $N \times (N-1)$ matrix. This matrix has $\frac{1}{2} (N^2 -  N)$ non-zero entries. We can construct this matrix with $\frac{1}{2} (N^2 -  N)$ elementary operations. 
\item $\boldsymbol{E}$: is a $(N-1) \times (N-1)$ matrix. This matrix has $(N-1)$ non-zero entries. We can construct this matrix with $\frac{1}{2} (N^2 -  N)$ elementary operations.  
\item $\boldsymbol{F}$: is a $N \times N$ matrix. This matrix has $N^2 - N$ non-zero entries. We can construct this matrix with $\frac{1}{2} (N^2 -  N)$ elementary operations.
\item $\boldsymbol{G}$: is a $N \times (N-1)$ matrix. This matrix has $\frac{1}{2} (N^2 -  N)$ non-zero entries. We can construct this matrix with $\frac{1}{2} (N^2 -  N)$ elementary operations.
\item $\boldsymbol{0}$: is a $(N-1) \times (N-1)$ all zero matrix. 
\end{enumerate}

We can construct $\boldsymbol{C}-\boldsymbol{G}$ in $O(N^2)$ 


The sparsity ratio of $\boldsymbol{H}$ is equal to
\begin{equation}
\gamma_N = \frac{10N^2-14N+6}{16N^2-16N+4} 
\end{equation}

The asymptotic sparsity ratio of $\boldsymbol{H}$ is equal to

\begin{equation}
\gamma = \lim_{N\rightarrow \infty}\frac{5N^2-7N+3}{8N^2-8N+2} = \frac{5}{8}
\end{equation}

Asymptotically the computational cost of the matrix-vector product between $\boldsymbol{H}$ and a vector of appropriate size is $O(P^2)$. However, the computational cost converges 
very slowly to its asymptotic value, and in general is equal to $O(P^{k_N})$, where $k_N < 2$.

\begin{eqnarray}
P^{k_N}_N &=& (1 - \gamma_N)P^2_N\\
k_N &=& \log_{P_N}(1 - \gamma_N) + 2\\
k &=& \lim_{N\rightarrow \infty} \log_{P_N}(1 - \gamma_N) + 2 = 2
\end{eqnarray}

Moreover,

\begin{eqnarray}
P^{c_N}_N &=& \frac{1}{4} P_N^2 -1\\
c_N &=& \log_{P_N}\left (\frac{1}{4} P_N^2-1 \right )\\
c &=& \lim_{N\rightarrow \infty} \log_{P_N}\left (\frac{1}{4} P_N^2-1\right ) = 2
\end{eqnarray}

NB $P_c$ IS WRONG HERE DID NOT TAKE INTO ACCOUNT THE CONJUGANTION OF $\boldsymbol{B}$ (need multiply its computational complexity by two).

\begin{figure}
  \includegraphics[width=\linewidth]{sparsity_factor.png}
  \caption{Sparsity factor $\gamma_N$.}
\end{figure}

\begin{figure}
  \includegraphics[width=\linewidth]{k_N.png}
  \caption{Computation complexity order of matrix-vector product and construction.}
\end{figure}

THE POTENTIAL GENERAL FORMULATION

Let $\phi_{pq}:\mathbb{N}^2\rightarrow\mathbb{N}$ be the mapping that maps antenna indices to the redundant spacing indices of the array. Also note that this mapping is not symmetric as 
$\phi_{pq} = 0$ if $p>q$. Moreover, let

\begin{equation}
\psi_{ij} = 
\begin{cases}
q~\textrm{if}~\exists! ~ q \in \mathbb{N} ~ s.t. ~(\phi_{iq} = j)\\
0~\textrm{otherwise}
\end{cases}
\end{equation},

\begin{equation}
\xi_{ij} = 
\begin{cases}
p~\textrm{if}~\exists! ~ p \in \mathbb{N} ~ s.t. ~(\phi_{pi} = j)\\
0~\textrm{otherwise}
\end{cases}
\end{equation}

and

\begin{equation}
\zeta_{ij} = 
\begin{cases}
\phi_{ij}~\textrm{if}~i<j\\
\phi_{ji}~\textrm{if}~i>j\\
0~\textrm{otherwise}
\end{cases}
\end{equation}





\begin{equation}
\boldsymbol{H} = 
\begin{bmatrix}
\boldsymbol{A} & \boldsymbol{B}\\
\boldsymbol{B}^* & \boldsymbol{A}^*
\end{bmatrix}
\end{equation}

With 

\begin{equation}
\boldsymbol{A} = 
\begin{bmatrix}
\boldsymbol{C} & \boldsymbol{D}\\
\boldsymbol{D^H} & \boldsymbol{E}
\end{bmatrix}
\end{equation}

\begin{equation}
\boldsymbol{B} = 
\begin{bmatrix}
\boldsymbol{F} & \boldsymbol{G}\\
\boldsymbol{G}^T & \boldsymbol{0}
\end{bmatrix}
\end{equation}

\begin{equation}
[\boldsymbol{C}]_{ij} = 
\begin{cases}
 \sum_{k \neq i} \left | g_k \right |^2 \left | y_{\zeta_{ik}} \right |^2 & \textrm{if} ~ i=j\\
 0 & \textrm{otherwise}
\end{cases}
\end{equation}

\begin{equation}
[\boldsymbol{D}]_{ij} = 
\begin{cases}
 g_i y_j^*  \left | g_{\psi_{ij}} \right |^2  & \textrm{if} ~ \psi_{ij}\neq0\\
 0 & \textrm{otherwise}
\end{cases}
\end{equation}

\begin{equation}
[\boldsymbol{E}]_{ij} = 
\begin{cases}
 \sum_{pq \in \mathcal{I}_i} \left | g_p \right |^2 \left | g_q \right |^2  & \textrm{if} ~ i=j\\
 0 & \textrm{otherwise}
\end{cases}
\end{equation}

\begin{equation}
[\boldsymbol{F}]_{ij} = 
\begin{cases}
 g_i g_j  \left | y_{\zeta_{ij}} \right |^2  & \textrm{if} ~ (i \neq j)\\
 0 & \textrm{otherwise}
\end{cases}
\end{equation}

\begin{equation}
[\boldsymbol{G}]_{ij} = 
\begin{cases}
 g_i y_j  \left | g_{\xi_{ij}} \right |^2  & \textrm{if} ~ (i > j)\wedge(\xi_{ij}\neq0)\\
 0 & \textrm{otherwise}
\end{cases}
\end{equation}

Moreover, 

\begin{equation}
\mathcal{I}_i = \left\{pq\in\mathbb{N}^2|(\phi_{pq} = i) \right\}.
\end{equation}

FOR AN EW ARRAY WE HAVE

\begin{equation}
\phi_{ij} = 
\begin{cases}
j - i & \textrm{if}~i<j\\
0 & \textrm{otherwise}
\end{cases}
\end{equation}

\begin{equation}
\psi_{ij} = 
\begin{cases}
i+j & \textrm{if}~(0<i<N)\wedge(0<j<N)\wedge(0< i+j \leq N)\\
0 & \textrm{otherwise}
\end{cases}
\end{equation}

\begin{equation}
\zeta_{ij} = 
\begin{cases}
j - i & \textrm{if}~i<j\\
i - j & \textrm{if}~i>j\\
0 & \textrm{otherwise}
\end{cases}
\end{equation}

\begin{equation}
\xi_{ij} = 
\begin{cases}
i-j & (0<i<N)\wedge(0<j<N)\wedge(i>j)\\
0 & \textrm{otherwise}
\end{cases}
\end{equation}





\end{section}
\end{document}
