\documentclass[a4paper,10pt]{article}
\usepackage[utf8]{inputenc}
\usepackage{amsmath,amsfonts,amssymb,graphicx,subfigure,mathtools}
\usepackage{amsmath}
\usepackage{rotating}
\usepackage{lscape}

\newcommand{\conj}[1]{\overline{#1}}

%MATRIX HELPER FUNCTIONS
\newcommand\coolover[2]{\mathrlap{\smash{\overbrace{\phantom{%
    \begin{matrix} #2 \end{matrix}}}^{\mbox{$#1$}}}}#2} 

\newcommand\coolunder[2]{\mathrlap{\smash{\underbrace{\phantom{%
    \begin{matrix} #2 \end{matrix}}}_{\mbox{$#1$}}}}#2}

\newcommand\coolleftbrace[2]{%
#1\left\{\vphantom{\begin{matrix} #2 \end{matrix}}\right.}

\newcommand\coolrightbrace[2]{%
\left.\vphantom{\begin{matrix} #1 \end{matrix}}\right\}#2}



%opening
\title{SPARC: SpArse Redundant Calibration}
\author{T.L. Grobler}

\begin{document}

\maketitle

%\begin{abstract}

%\end{abstract}

\begin{section}{Alternative derivation of Eq.}
The basic StEfCal update step is equal to Eq. %\citet{Salvini2015}.
Assume without any loss of generality that the array is in an east west regular grid. Furthermore, assume that $\boldsymbol{d}$ has been re-ordered to
\begin{equation}
\widetilde{\boldsymbol{d}} = \left[d{12},\cdots,d_{N-1,N},d_{13},\cdots,d_{N-2,N},\cdots,d_{1N}\right]^T 
\end{equation}

\begin{equation}
\widetilde{\boldsymbol{d}} = 
\begin{bmatrix} 
d{12}\\
\vdots\\
d_{N-1,N}\\
d_{13}\\
\vdots\\
d_{N-2,N}\\
\vdots\\
d_{1N}
\end{bmatrix} 
\end{equation}.
Eq.~\eqref{} can now be rewritten as 
\begin{equation}
\widetilde{\boldsymbol{d}} = \boldsymbol{R}\boldsymbol{y}, 
\end{equation}
if we assume that $\boldsymbol{g}$ and its conjugate are known vectors. In Eq.~\eqref{},
\begin{equation}
\boldsymbol{R} = 
\begin{bmatrix}
g_1\conj{g}_2 & 0 & \cdots & 0\\
g_2\conj{g}_3 & 0 & \cdots & 0\\
\vdots & 0 & \cdots & 0\\
g_{N-1}\conj{g}_N & 0 & \cdots & 0\\
0 & g_1\conj{g}_3 & \cdots & 0\\
0 & \vdots & \cdots & 0\\
0 & g_{N-2}\conj{g}_N & \cdots & 0\\
0 & 0 & \cdots & 0\\
  & \vdots & \\
0 & 0 & \cdots & g_1\conj{g}_N\\  
\end{bmatrix}
\end{equation}
We can now determine $\boldsymbol{y}$ with
\begin{equation}
\boldsymbol{y} = (\boldsymbol{R}^H\boldsymbol{R})^{-1}\boldsymbol{R}^H\widehat{\boldsymbol{v}}, 
\end{equation}
where 
\begin{equation}
[\boldsymbol{R}^H\boldsymbol{R}]_{ij} = 
\begin{cases}
\sum_{rs\in\mathcal{RS}_i} |g_r|^2|g_s|^2 &\textrm{if}~i=j\\
0&\textrm{otherwise}
\end{cases}
\end{equation}
and
\begin{equation}
[\boldsymbol{R}^H\widehat{\boldsymbol{v}}]_i = \sum_{rs\in\mathcal{RS}_i} \conj{g}_r g_s d_{rs} 
\end{equation}
Substituting Eq and Eq into Eq we obtain
the leftmost term in Eq (barring $\alpha$ of course).  
\end{section}
\begin{section}{Proof of Eq...}
We have that
\begin{equation}
\lim_{N\rightarrow\infty} P^{\log_{P}(1-\gamma)} = 1-\gamma_{\infty},
\end{equation}
which is only possible if 
\begin{equation}
\lim_{N\rightarrow\infty} \log_{P}(1-\gamma) = 0,
\end{equation}
implying that $c_{\infty} = 2$.

 
\end{section}
\end{document}
 